\documentclass{report}

\usepackage[T2A]{fontenc}
\usepackage[utf8]{luainputenc}
\usepackage[english, russian]{babel}
\usepackage[pdftex]{hyperref}
\usepackage[14pt]{extsizes}
\usepackage{listings}
\usepackage{color}
\usepackage{geometry}
\usepackage{enumitem}
\usepackage{multirow}
\usepackage{graphicx}
\usepackage{indentfirst}

\geometry{a4paper,top=2cm,bottom=3cm,left=2cm,right=1.5cm}
\setlength{\parskip}{0.5cm}
\setlist{nolistsep, itemsep=0.3cm,parsep=0pt}

\lstset{language=C++,
basicstyle=\footnotesize,
keywordstyle=\color{blue}\ttfamily,
stringstyle=\color{red}\ttfamily,
commentstyle=\color{green}\ttfamily,
morecomment=[l][\color{magenta}]{\#},
tabsize=4,
breaklines=true,
breakatwhitespace=true,
title=\lstname,
}

\makeatletter
\renewcommand\@biblabel[1]{#1.\hfil}
\makeatother

\begin{document}

\begin{titlepage}

\begin{center}
Министерство науки и высшего образования Российской Федерации
\end{center}

\begin{center}
Федеральное государственное автономное образовательное учреждение высшего образования \\
Национальный исследовательский Нижегородский государственный университет им. Н.И. Лобачевского
\end{center}

\begin{center}
Институт информационных технологий, математики и механики
\end{center}

\vspace{4em}

\begin{center}
\textbf{\LargeОтчет по лабораторной работе} \\
\end{center}
\begin{center}
\textbf{\Large«Интегрирование многомерных интеграллов методом Монте-Карло»} \\
\end{center}

\vspace{4em}

\newbox{\lbox}
\savebox{\lbox}{\hbox{text}}
\newlength{\maxl}
\setlength{\maxl}{\wd\lbox}
\hfill\parbox{7cm}{
\hspace*{5cm}\hspace*{-5cm}\textbf{Выполнила:} \\ студентка группы 381908-4 \\ Андрич Мария\\
\\
\hspace*{5cm}\hspace*{-5cm}\textbf{Проверил:}\\ доцент кафедры МОСТ, \\ кандидат технических наук \\ Сысоев А.В.\\
\vspace{\fill}

\begin{center} Нижний Новгород \\ 2022 \end{center}

\end{titlepage}

\setcounter{page}{2}

% Содержание
\tableofcontents
\newpage

% Введение
\section*{Введение}
Метод Монте-Карло — это группа численных методов, основанных на
получении большого количества реализаций случайного процесса, который
формируется таким образом, чтобы его вероятностные характеристики
совпадали с аналогичными величинами решаемой задачи.
\par Для вычисления интегралов на компьютере используются так называемые численные методы, позволяющие находить решение задачи с заданной точностью.
\par Моделирование по методу Монте-Карло дает гораздо более полное
представление о возможных событиях. Оно позволяет судить не только о том,
что может произойти, но и о том, какова вероятность того или иного исхода.
Характер полученных данных при использовании метода позволяет создавать
графики различных последствий, а также вероятностей их наступления.
Преимущество метода в том, что после финальных расчетов можно
увидеть, какие факторы оказывают наибольшее воздействие на итоговые
результаты.
\addcontentsline{toc}{section}{Введение}

\newpage

% Постановка задачи
\section*{Постановка задачи}
\addcontentsline{toc}{section}{Постановка задачи}
В рамках лабораторной работы необходимо:
\begin{enumerate}
\item Изучить теоретические основы метода Монте-Карло.
\item Реализовать последовательную и параллельные реализации алгоритма. вычисления многомерных интегралов методом Монте-Карло.
\item Проверить корректность работы алгоритмов.
\item Провести эксперименты для оценки эффективности параллелизации.
\item Сделать соответвующие выводы на основе полученных результатов.
\end{enumerate}
\par Для реализации параллельных версий необходимо использовать библиотеки OpenMP, Intel Threading Building Blocks и std::threads. Для проверки корректности работы использовать Google Testing Framework.
\newpage

% Описание алгоритмов
\section*{Описание алгоритмов}
\addcontentsline{toc}{section}{Описание алгоритмов}
Предположим, необходимо взять интеграл от некоторой функции. Воспользуемся неформальным геометрическим описанием интеграла и будем понимать его как площадь под графиком этой функции.
\par Пусть имеется интеграл: $\int\limits_a^b f(x)dx$ , где $$ a\le x \le b $$

Рассмотрим случайную величину u, равномерно распределённую на отрезке интегрирования [a,b]. Тогда $f(u)$ также будет случайной величиной, причём её математическое ожидание выражается как
$$E f(u)=\int\limits_{a}^{b}f(x)\varphi(x)dx$$
где $\varphi(x)$ — плотность распределения случайной величины u, равная 
$\frac {1} {b-a}$ на участке [a,b]. Таким образом, искомый интеграл выражается как
$$\int\limits_a^b f(x)dx = (b-a) E f(u),$$
но математическое ожидание случайной величины $f(u)$ можно легко оценить, смоделировав эту случайную величину и посчитав выборочное среднее.

Далее случайно выбираются N точек, равномерно распределённых на [a,b], для каждой точки $u_i$ вычисляем $f(u_i)$. Затем вычисляем выборочное среднее: 
$$\frac{1}{N}\sum _{{i=1}}^{{N}}f(u_{i})$$

В итоге получаем оценку интеграла:

$$\int \limits _a^bf(x)\,dx\approx \frac {b-a}{N}\sum _{i=1}^{N}f(u_{i})$$

Точность оценки зависит только от количества точек N.
\par В случае вычисления многомерных интегралов формула никак не изменяется, просто вместо одномерного интеграла используется многомерный. Тогда проход проводится по параллелепипедам или другим многомерным объектам.
\newpage

% Схема распараллеливания
\section*{Схема распараллеливания}
\addcontentsline{toc}{section}{Схема распараллеливания}
\par Идея параллельной функции метода Монте-Карло состоит в том, что итерации цикла делятся между потоками, после разбиения интеграла каждый поток получает одинаковое количество итераций, эти потоки проводят вычисления в своём диапазоне, далее складывается результат их работы.
\newpage

% Описание программной реализации
\section*{Описание программной реализации}
\addcontentsline{toc}{section}{Описание программной реализации}
\par Последовательный алгоритм многомерного интегрирования методом Монте-Карло вызывается функцией:
\begin{lstlisting}
double seqMonteCarlo(double(*f)(const std::vector<double>&),
                                const std::vector<double>& a,
                                const std::vector<double>& b, int steps);
\end{lstlisting}
\par Где входными параметрами функции являются: интегрируемая функция, которая выбрана для вычисления интеграла, следующими параметрами являются границы отрезков a и b, последним параметром является частота разбиения.
\subsection*{OpenMP}
\addcontentsline{toc}{subsection}{OpenMP}
\par Параллельный алгоритм OpenMP вызывается функцией:
\begin{lstlisting}
double ompMonteCarlo(double(*f)(const std::vector<double>&),
                                const std::vector<double>& a,
                                const std::vector<double>& b, int steps);
\end{lstlisting}
\par Распараллеливание с помощью директивы:
\par\verb|#pragma omp parallel shared(r) reduction(+ : res)|
\subsection*{TBB}
\addcontentsline{toc}{subsection}{TBB}
\par Параллельный алгоритм TBB:
\begin{lstlisting}
double tbbMonteCarlo(double(*f)(const std::vector<double>&),
                                const std::vector<double>& a,
                                const std::vector<double>& b, int steps);
\end{lstlisting}
\par Распараллеливание ведётся через операцию редукии: \verb| tbb::parallel_reduce|\par На вход поступает \verb|tbb::blocked_range<size_t>(0, steps)|
\subsection*{std::threads}
\addcontentsline{toc}{subsection}{std::threads}
\begin{lstlisting}
double stdMonteCarlo(double(*f)(const std::vector<double>&),
                                const std::vector<double>& a,
                                const std::vector<double>& b, int steps)
\end{lstlisting}
\par \verb|std::thread::hardware_concurrency();| - возвращает количество параллельных потоков, поддерживаемых реализацией. Итерации делятся между потоками. 
\par В \verb|threads.emplace_back([&,range,i]();| генерируется случайное разбиение фунции. Далее функция join() ожидает завершения работы всех потоков и возвращает результат.

\par Входные параметры функции для всех параллельных реализаций идентичны последовательной.

\newpage

% Подтверждение корректности
\section*{Подтверждение корректности}
\addcontentsline{toc}{section}{Подтверждение корректности}
Для подтверждения корректности программы представлен набор тестов, разработанных с использованием Google Testing Framework.
\par Тесты подразумевают вычисление интеграла для заранее заданной функции параллельным и последовательным способом, также проводится замер времени затраченного на получение результата.
\par Также представлен тест для определения отрицательных значений разбиений в интеграле.
\par Для успешного прохождения теста требуется, чтобы отклонение между последовательной и параллельной версиями находилось в заданных тестом рамках. \par Успешное прохождение всех тестов доказывает корректность работы данных реализаций.
\newpage

% Результаты экспериментов
\section*{Результаты экспериментов}
\addcontentsline{toc}{section}{Результаты экспериментов}
Эксперименты для оценки эффективности проводились на ПК со следующими характеристиками:

\begin{itemize}
\item Процессор: AMD А6;
\item Оперативная память: 4 ГБ ;
\item ОС: Microsoft Windows 10 ;
\end{itemize}
\par Замеры параллельных версий производились на 4 потоках.
\begin{table}[!h]
\caption{Результаты экспериментов}
\centering
\begin{tabular}{lllll}
Частота разбиения & Последовательная & OpenMP & TBB & std::threads     \\
100000    & 0.07                    & 0.02       & 0.004    & 0.03     \\
150000    & 0.11                    & 0.03       & 0.007    & 0.04     \\
200000   & 0.15                    & 0.04       & 0.01      & 0.04     \\
250000   & 0.17                   & 0.05       & 0.01       & 0.05     \\
\end{tabular}
\end{table}

\par Можно сделать вывод о том, что параллельная реализация работает быстрее, чем последовательная, причём ускорение более заметно на большем числе разбиений.
\par Самой эффективной оказалась реализация с использованием TBB технологий.
\newpage

% Заключение
\section*{Заключение}
\addcontentsline{toc}{section}{Заключение}
В ходе выполнения данной лабораторной работы были реализованы последовательная и параллельная реализации алгоритма вычисления многомерных интегралов методом Монте-Карло.
\par Параллельные реализации доказали свою эффективность, благодаря использованию тестов, созданных для данной программы с использованием Google C++ TestingFramework.
\newpage
% Список литературы
\begin{thebibliography}{1}
\addcontentsline{toc}{section}{Список литературы}
\bibitem {wikishell} Википедия:https://ru.wikipedia.org/wiki/%D0%9C%D0%B5%D1%82%D0%BE%D0%B4_%D0%9C%D0%BE%D0%BD%D1%82%D0%B5-%D0%9A%D0%B0%D1%80%D0%BB%D0%BE
\bibitem {Sobol} Соболь И.М. Популярные лекции по математике 1968. Выпуск 46.
Метод Монте-Карло. М.: Наука, 1968. — 64 с.
\end{thebibliography}
\newpage

% Приложение
\section*{Приложение}
\addcontentsline{toc}{section}{Приложение}
\par Последовательная версия
\par \verb|1.monte_carlo.cpp|
\begin{lstlisting}
// Copyright 2022 Andrich Maria
#include <vector>
#include <string>
#include <random>
#include <algorithm>
#include <ctime>
#include "../../modules/task_1/andrich_m_montecarlo/montecarlo.h"

double seqMonteCarlo(double(*f)(std::vector<double>), const std::vector<double>& a,
    const std::vector<double>& b, int s) {
    double res = 0.0;
    std::mt19937 gen;
    gen.seed(static_cast<unsigned int>(time(0)));
    if (s <= 0)
        throw "integral is negative";
    int mult = a.size();
    double S = 1;
    for (int i = 0; i < mult; i++)
        S *= (b[i] - a[i]);
    std::vector<std::uniform_real_distribution<double>> r(mult);
    std::vector<double> r1(mult);
    for (int i = 0; i < mult; i++)
        r[i] = std::uniform_real_distribution<double>(a[i], b[i]);
    for (int i = 0; i < s; i++) {
        for (int j = 0; j < mult; j++)
            r1[j] = r[j](gen);
        res += f(r1);
    }
    res *= S / s;
    return res;
}
\end{lstlisting}
\par \verb|2.montecarlo.h|
\begin{lstlisting}
// Copyright 2022 Andrich Maria
#ifndef MODULES_TASK_1_ANDRICH_M_MONTECARLO_MONTECARLO_H_
#define MODULES_TASK_1_ANDRICH_M_MONTECARLO_MONTECARLO_H_

#include <vector>
#include <string>

double seqMonteCarlo(double(*f)(std::vector<double>), const std::vector<double>&a,
    const std::vector<double>&b, int s);

#endif  // MODULES_TASK_1_ANDRICH_M_MONTECARLO_MONTECARLO_H_
\end{lstlisting}
\par \verb|3.main.cpp|
\begin{lstlisting}
// Copyright 2022 Andrich Maria
#include <gtest/gtest.h>
#include <vector>
#include <cmath>
#include "./montecarlo.h"

double seqIntegral_1(std::vector<double> x) {
    return x[0] * x[0];
}

double seqIntegral_2(std::vector<double> x) {
    return 3 * x[0] * x[0] * x[0] + 2 * x[1] * x[1];
}

double seqIntegral_3(std::vector<double> x) {
    return sin(x[0]) + 2 * x[1] + x[2] * x[2];
}

double seqIntegral_4(std::vector<double> x) {
    return x[0] * x[0] + 2 * x[1] - cos(x[2]) + 2 * x[3] * x[3] * x[3] - 3 * x[4];
}


TEST(Monte_carlo_integral_test, test_result_of_integral) {
    double a = 0.0, b = 1.0;
    double res = seqMonteCarlo(seqIntegral_1, { a }, { b }, 1000000);
    ASSERT_NEAR(res, 0.333, 0.5);
}

TEST(Monte_carlo_integral_test, test_result_of_integral_1) {
    std::vector<double>a = { 0.0, 2.5 };
    std::vector<double>b = { 1.534, 3.12 };
    double res = seqMonteCarlo(seqIntegral_2, a, b, 1000000);
    ASSERT_NEAR(res, 17.660, 0.5);
}

TEST(Monte_carlo_integral_test, test_result_of_integral_2) {
    std::vector<double>a = { 0.0, 2.5, 1.234 };
    std::vector<double>b = { 1.534, 3.12, 1.555 };
    double res = seqMonteCarlo(seqIntegral_3, a, b, 1000000);
    ASSERT_NEAR(res, 2.503, 0.5);
}

TEST(Monte_carlo_integral_test, test_result_of_integral_3) {
    std::vector<double>a = { 0.3, 3.5, 1.234 };
    std::vector<double>b = { 3.534, 1.32, 1.435 };
    double res = seqMonteCarlo(seqIntegral_3, a, b, 1000000);
    ASSERT_NEAR(res, -10.181, 0.5);
}

TEST(Monte_carlo_integral_test, test_n_is_negative) {
    double a = 0.0, b = 3.0;
    ASSERT_ANY_THROW(seqMonteCarlo(seqIntegral_1, { a }, { b }, -1000));
}

int main(int argc, char** argv) {
    ::testing::InitGoogleTest(&argc, argv);
    return RUN_ALL_TESTS();
}
\end{lstlisting}

\par OpenMP Версия
\par \verb|1.monte_carlo.cpp|
\begin{lstlisting}
#include <iostream>
#include "../../../modules/task_2/andric_m_montecarlo/montecarlo.h"

double MonteCarloSeq(double(*func)(const std::vector<double>&),
    const std::vector<double>& left, const std::vector<double>& right,
    uint64_t steps, int seed) {
    // CHECK DIMENSIONS
    if (left.size() != right.size())
        throw "Different sizes of left and right borders";
    if (left.empty())
        throw "Empty borders";

    double res = 0.0;
    size_t dimension = left.size();

    // Random
    std::mt19937 gen;
    gen.seed(seed);
    std::vector<std::uniform_real_distribution<double>> realdistr(dimension);
    for (size_t i = 0; i < dimension; i++)
        realdistr[i] = std::uniform_real_distribution<double>(left[i],
                                                              right[i]);

    // Main cycle
    std::vector<double> pnt(dimension);
    for (uint64_t i = 0; i < steps; ++i) {
        for (size_t k = 0; k < dimension; ++k) {
            pnt[k] = realdistr[k](gen);
        }
        res += func(pnt);
    }

    // Area
    double S = 1;
    for (size_t k = 0; k < dimension; ++k) {
        S *= right[k] - left[k];
    }
    res = S * res / steps;
    return res;
}

double MonteCarloOmp(double(*func)(const std::vector<double>&),
    const std::vector<double>& left, const std::vector<double>& right,
    uint64_t steps, int threadnum, int seed) {
    if (left.size() != right.size())
        throw "Different sizes of left and right borders";
    if (left.empty())
        throw "Empty borders";

    double res = 0.0;
    size_t dimension = left.size();

    // Random
    std::mt19937 gen;
    gen.seed(seed);
    std::vector<std::uniform_real_distribution<double>> realdistr(dimension);
    for (size_t i = 0; i < dimension; i++)
        realdistr[i] = std::uniform_real_distribution<double>(left[i],
            right[i]);
    // Main cycle
    std::vector<double> pnt(dimension);
    #pragma omp parallel firstprivate(pnt) num_threads(threadnum) reduction(+ : res)
    {
        for (uint64_t i = 0; i < steps; ++i) {
            for (size_t k = 0; k < dimension; ++k) {
                pnt[k] = realdistr[k](gen);
            }
            res += func(pnt);
        }
    }

    // Area
    double S = 1;
    for (size_t k = 0; k < dimension; ++k) {
        S *= right[k] - left[k];
    }
    res = S * res / (steps * threadnum);
    return res;
}

\end{lstlisting}
\par \verb|2.monte_carlo.h|
\begin{lstlisting}
#ifndef MODULES_TASK_2_ANDRIC_M_MONTECARLO_MONTECARLO_H_
#define MODULES_TASK_2_ANDRIC_M_MONTECARLO_MONTECARLO_H_

#include <omp.h>

#include <random>

#include <vector>

double MonteCarloSeq(double(*func)(const std::vector < double > &),
  const std::vector < double > & left,
    const std::vector < double > & right,
      uint64_t steps, int seed = 0);

double MonteCarloOmp(double(*func)(const std::vector < double > &),
  const std::vector < double > & left,
    const std::vector < double > & right,
      uint64_t steps, int threadnum, int seed = 0);

#endif  // MODULES_TASK_2_ANDRIC_M_MONTECARLO_MONTECARLO_H_
\end{lstlisting}
\par \verb|3.main.cpp|
\begin{lstlisting}
#include <gtest/gtest.h>

#include <iostream>

#include <vector>

#include <algorithm>

#include "./montecarlo.h"

double test_func1(const std::vector < double > & x) {
  return 1;
}

double test_func2(const std::vector < double > & x) {
  return x[0];
}

double test_func3(const std::vector < double > & x) {
  return x[0] + x[1];
}

double test_func4(const std::vector < double > & x) {
  double res = 0;
  res += cos(x[0] * x[1]) + sin(x[0] + x[1]) * 5 + x[0] - sqrt(abs(x[1]));
  return res;
}

TEST(MonteCarlo_test_omp, test_different_sized_borders) {
  std::vector < double > left;
  std::vector < double > right;
  const int thread_num = 4;
  ASSERT_ANY_THROW(MonteCarloOmp(test_func3, left, right, 1000, thread_num));
}

TEST(MonteCarlo_test_omp, test_empty_borders) {
  std::vector < double > left;
  std::vector < double > right;
  const int thread_num = 4;
  ASSERT_ANY_THROW(MonteCarloOmp(test_func3, left, right, 1000, thread_num));
}

TEST(MonteCarlo_test_omp, test_1) {
  std::vector < double > left = {
    0.0
  };
  std::vector < double > right = {
    5.0
  };
  const uint64_t steps = 10000;
  const int thread_num = 4;

  double res_omp = 0.0;
  ASSERT_NO_THROW(res_omp = MonteCarloOmp(test_func1, left, right, steps, thread_num));
  double res_seq = 0.0;
  ASSERT_NO_THROW(res_seq = MonteCarloSeq(test_func1, left, right, steps * thread_num));
  std::cout << res_omp << " " << res_seq << "\n";
  ASSERT_NEAR(res_omp, 5, 0.01);
}

TEST(MonteCarlo_test_omp, test_2) {
  std::vector < double > left = {
    0.0
  };
  std::vector < double > right = {
    2.0
  };
  const uint64_t steps = 1000;

  double res_omp = 0.0;
  ASSERT_NO_THROW(res_omp = MonteCarloOmp(test_func2, left, right, steps, 1));
  double res_seq = 0.0;
  ASSERT_NO_THROW(res_seq = MonteCarloSeq(test_func2, left, right, steps * 1));
  std::cout << res_omp << " " << res_seq << "\n";
}

TEST(MonteCarlo_test_omp, test_3) {
  std::vector < double > left = {
    0.0,
    0.0
  };
  std::vector < double > right = {
    2.0,
    2.0
  };
  const uint64_t steps = 10000;
  const int thread_num = 4;

  double res_omp = 0.0;
  ASSERT_NO_THROW(res_omp = MonteCarloOmp(test_func3, left, right, steps, thread_num));
  double res_seq = 0.0;
  ASSERT_NO_THROW(res_seq = MonteCarloSeq(test_func3, left, right, steps * thread_num));

  std::cout << res_omp << " " << res_seq << "\n";
}
\end{lstlisting}

\par Intel TBB версия
\par \verb|1.monte_carlo.cpp|
\begin{lstlisting}
// Copyright 2022 Andrich Maria
#include <tbb/tbb.h>
#include <functional>
#include <ctime>
#include <random>
#include <string>
#include <algorithm>
#include <vector>
#include "../../../modules/task_3/andrich_m_montecarlo/montecarlo.h"

double seqMonteCarlo(double(*f)(const std::vector<double>&),
                                const std::vector<double>& a,
                                const std::vector<double>& b, int steps) {
    if (steps <= 0)
        throw "integral is negative";
    double res = 0.0;
    std::mt19937 gen;
    gen.seed(static_cast<unsigned int>(time(0)));

    int mult = a.size();
    double S = 1;
    for (int i = 0; i < mult; i++)
        S *= (b[i] - a[i]);

    std::vector<std::uniform_real_distribution<double>> r(mult);
    std::vector<double> r1(mult);
    for (int i = 0; i < mult; i++)
        r[i] = std::uniform_real_distribution<double>(a[i], b[i]);

    for (int i = 0; i < steps; ++i) {
        for (int j = 0; j < mult; ++j)
            r1[j] = r[j](gen);
        res += f(r1);
    }

    res *= S / steps;
    return res;
}

double tbbMonteCarlo(double(*f)(const std::vector<double>&),
                                const std::vector<double>& a,
                                const std::vector<double>& b, int steps) {
    if (steps <= 0)
        throw "integral is negative";
    double res = 0.0;
    int mult = a.size();
    std::vector<std::uniform_real_distribution<double>> r(mult);
    for (int i = 0; i < mult; i++)
        r[i] = std::uniform_real_distribution<double>(a[i], b[i]);

    res = tbb::parallel_reduce(
                tbb::blocked_range<size_t>(0, steps), 0.0,
                [&](tbb::blocked_range<size_t> range, double running_total) {
                    std::mt19937 gen;
                    gen.seed(static_cast<unsigned int>(time(0)));
                    std::vector<double> r1(mult);
                    for (size_t i = range.begin(); i != range.end(); ++i) {
                        for (int j = 0; j < mult; ++j)
                            r1[j] = r[j](gen);
                        running_total += f(r1);
                    }

                    return running_total;
                }, std::plus<double>() );

    double S = 1;
    for (int i = 0; i < mult; i++)
        S *= (b[i] - a[i]);
    res *= S / steps;

    return res;
}
\end{lstlisting}
\par \verb|2.monte_carlo.h|
\begin{lstlisting}
// Copyright 2022 Andrich Maria
#ifndef MODULES_TASK_3_ANDRICH_M_MONTECARLO_MONTECARLO_H_
#define MODULES_TASK_3_ANDRICH_M_MONTECARLO_MONTECARLO_H

#include <vector>
#include <string>

double seqMonteCarlo(double(*f)(const std::vector<double>&),
                     const std::vector<double>& a,
                     const std::vector<double>& b, int s);

double tbbMonteCarlo(double(*f)(const std::vector<double>&),
                     const std::vector<double>& a,
                     const std::vector<double>& b, int s);

#endif  // MODULES_TASK_3_ANDRICH_M_MONTECARLO_MONTECARLO_H
\end{lstlisting}
\par \verb|3.main.cpp|
\begin{lstlisting}
// Copyright 2022 Andrich Maria
#include <gtest/gtest.h>
#include <tbb/tick_count.h>
#include <cmath>
#include <vector>
#include "../../../modules/task_3/andrich_m_montecarlo/montecarlo.h"

double Integral_1(const std::vector<double>& x) {
    return x[0] * x[0];
}

double Integral_2(const std::vector<double>& x) {
    return 3 * x[0] * x[0] * x[0] + 2 * x[1] * x[1];
}

double Integral_3(const std::vector<double>& x) {
    return sin(x[0]) + 2 * x[1] + x[2] * x[2];
}

double Integral_4(const std::vector<double>& x) {
    return x[0] * x[0] + 2 * x[1] - cos(x[2]) +
            2 * x[3] * x[3] * x[3] - 3 * x[4];
}


TEST(Monte_carlo_integral_test, test_result_of_integral) {
    double a = 0.0, b = 1.0;
    tbb::tick_count t1, t2;
    int num_th = 4;
    const int N = 250000;
    t1 = tbb::tick_count::now();
    double res_seq = seqMonteCarlo(Integral_1, { a }, { b }, N * num_th);
    t2 = tbb::tick_count::now();
    std::cout << " Seq_Time = " << (t2 - t1).seconds() << std::endl;
    t1 = tbb::tick_count::now();
    double res_tbb = tbbMonteCarlo(Integral_1, { a }, { b }, N);
    t2 = tbb::tick_count::now();
    std::cout << " Paral_Time = " << (t2 - t1).seconds() << std::endl;

    ASSERT_NEAR(res_seq, res_tbb, 0.5);
}

TEST(Monte_carlo_integral_test, test_result_of_integral_1) {
    std::vector<double> a = { 0.0, 2.5 };
    std::vector<double> b = { 1.534, 3.12 };
    tbb::tick_count t1, t2;
    int num_th = 4;
    const int N = 250000;

    t1 = tbb::tick_count::now();
    double res_seq = seqMonteCarlo(Integral_2, a, b, N * num_th);
    t2 = tbb::tick_count::now();
    std::cout << " Seq_Time = " << (t2 - t1).seconds() << std::endl;
    t1 = tbb::tick_count::now();
    double res_tbb = tbbMonteCarlo(Integral_2, a, b, N);
    t2 = tbb::tick_count::now();
    std::cout << " Paral_Time = " << (t2 - t1).seconds() << std::endl;

    ASSERT_NEAR(res_seq, res_tbb, 0.5);
}

TEST(Monte_carlo_integral_test, test_result_of_integral_2) {
    std::vector<double> a = { 0.0, 2.5, 1.234 };
    std::vector<double> b = { 1.534, 3.12, 1.555 };
    tbb::tick_count t1, t2;
    int num_th = 4;
    const int N = 250000;

    t1 = tbb::tick_count::now();
    double res_seq = seqMonteCarlo(Integral_3, a, b, N * num_th);
    t2 = tbb::tick_count::now();
    std::cout << " Seq_Time = " << (t2 - t1).seconds() << std::endl;
    t1 = tbb::tick_count::now();
    double res_tbb = tbbMonteCarlo(Integral_3, a, b, N);
    t2 = tbb::tick_count::now();
    std::cout << " Paral_Time = " << (t2 - t1).seconds() << std::endl;

    ASSERT_NEAR(res_seq, res_tbb, 0.5);
}

TEST(Monte_carlo_integral_test, test_result_of_integral_3) {
    std::vector<double>a = { 0.3, 1.32, 1.234 };
    std::vector<double>b = { 3.534, 3.5, 1.435 };
    tbb::tick_count t1, t2;
    int num_th = 4;
    const int N = 250000;

    t1 = tbb::tick_count::now();
    double res_seq = seqMonteCarlo(Integral_3, a, b, N * num_th);
    t2 = tbb::tick_count::now();
    std::cout << " Seq_Time = " << (t2 - t1).seconds() << std::endl;
    t1 = tbb::tick_count::now();
    double res_tbb = tbbMonteCarlo(Integral_3, a, b, N);
    t2 = tbb::tick_count::now();
    std::cout << " Paral_Time = " << (t2 - t1).seconds() << std::endl;

    ASSERT_NEAR(res_seq, res_tbb, 0.5);
}

TEST(Monte_carlo_integral_test, test_n_is_negative) {
    double a = 0.0, b = 3.0;
    ASSERT_ANY_THROW(tbbMonteCarlo(Integral_1, { a }, { b }, -1000));
}

int main(int argc, char** argv) {
    ::testing::InitGoogleTest(&argc, argv);
    return RUN_ALL_TESTS();
}
\end{lstlisting}

\par std::threads версия
\par \verb|1.monte_carlo.cpp|
\begin{lstlisting}
// Copyright 2022 Andrich Maria
#include <functional>
#include <ctime>
#include <random>
#include <string>
#include <algorithm>
#include <iostream>
#include <vector>
#include "../../3rdparty/unapproved/unapproved.h"
#include "../../../modules/task_4/andrich_m_montecarlo/montecarlo.h"

double seqMonteCarlo(double(*f)(const std::vector<double>&),
                                const std::vector<double>& a,
                                const std::vector<double>& b, int steps) {
    if (steps <= 0)
        throw "integral is negative";
    double res = 0.0;
    std::mt19937 gen;
    gen.seed(static_cast<unsigned int>(time(0)));

    int mult = a.size();
    double S = 1;
    for (int i = 0; i < mult; i++)
        S *= (b[i] - a[i]);

    std::vector<std::uniform_real_distribution<double>> r(mult);
    std::vector<double> r1(mult);
    for (int i = 0; i < mult; i++)
        r[i] = std::uniform_real_distribution<double>(a[i], b[i]);

    for (int i = 0; i < steps; ++i) {
        for (int j = 0; j < mult; ++j)
            r1[j] = r[j](gen);
        res += f(r1);
    }

    res *= S / steps;
    return res;
}

double stdMonteCarlo(double(*f)(const std::vector<double>&),
                                const std::vector<double>& a,
                                const std::vector<double>& b, int steps) {
    if (steps <= 0)
        throw "integral is negative";
    double res = 0.0;
    auto mult = a.size();
    std::vector<std::uniform_real_distribution<double>> r(mult);
    for (size_t i = 0; i < mult; i++)
        r[i] = std::uniform_real_distribution<double>(a[i], b[i]);

    auto num_threads = std::thread::hardware_concurrency();
    std::vector<std::thread> threads;
    std::vector<double> local_res(num_threads);
    auto start = 0;
    size_t k = steps % num_threads;

    for (size_t i = 0; i < num_threads; ++i) {
        my_range range;
        if (k > i) {
            range.start = start;
            range.end = start + (steps / num_threads) + 1;
            start += (steps / num_threads) + 1;
        } else {
            range.start = start;
            range.end = start + (steps / num_threads);
            start += (steps / num_threads);
        }
        threads.emplace_back([&, range, i]() {
            std::mt19937 gen;
            gen.seed(static_cast<unsigned int>(time(0)));
            std::vector<double> r1(mult);
            local_res[i] = 0;
            for (int k = range.start; k < range.end; ++k) {
                for (size_t j = 0; j < mult; ++j)
                    r1[j] = r[j](gen);
                local_res[i] += f(r1);
            }
        });
    }

    for (auto &thread : threads) {
        thread.join();
    }

    for (size_t i = 0; i < num_threads; ++i) {
        res += local_res[i];
    }

    double S = 1;
    for (size_t i = 0; i < mult; i++)
        S *= (b[i] - a[i]);
    res *= S / steps;

    return res;
}
\end{lstlisting}
\par \verb|2.monte_carlo.h|
\begin{lstlisting}
// Copyright 2022 Andrich Maria
#ifndef MODULES_TASK_4_ANDRICH_M_MONTECARLO_MONTECARLO_H_
#define MODULES_TASK_4_ANDRICH_M_MONTECARLO_MONTECARLO_H_

#include <vector>
#include <string>

struct my_range{
    int start;
    int end;
};

double seqMonteCarlo(double(*f)(const std::vector<double>&),
                     const std::vector<double>& a,
                     const std::vector<double>& b, int s);

double stdMonteCarlo(double(*f)(const std::vector<double>&),
                     const std::vector<double>& a,
                     const std::vector<double>& b, int s);

#endif  // MODULES_TASK_4_ANDRICH_M_MONTECARLO_MONTECARLO_H_
\end{lstlisting}
\par \verb|3.main.cpp|
\begin{lstlisting}
// Copyright 2022 Andrich Maria
#include <gtest/gtest.h>
#include <cmath>
#include <vector>
#include <chrono>
#include "../../../modules/task_4/andrich_m_montecarlo/montecarlo.h"

double Integral_1(const std::vector<double>& x) {
    return x[0] * x[0];
}

double Integral_2(const std::vector<double>& x) {
    return 3 * x[0] * x[0] * x[0] + 2 * x[1] * x[1];
}

double Integral_3(const std::vector<double>& x) {
    return sin(x[0]) + 2 * x[1] + x[2] * x[2];
}

double Integral_4(const std::vector<double>& x) {
    return x[0] * x[0] + 2 * x[1] - cos(x[2]) +
            2 * x[3] * x[3] * x[3] - 3 * x[4];
}


TEST(Monte_carlo_integral_test, test_result_of_integral) {
    double a = 0.0, b = 1.0;
    int num_th = 4;
    const int N = 250000;

    auto t1 = std::chrono::high_resolution_clock::now();
    double res_seq = seqMonteCarlo(Integral_1, { a }, { b }, N * num_th);
    auto t2 = std::chrono::high_resolution_clock::now();
    auto timeSeq = std::chrono::duration_cast<std::chrono::milliseconds>
        (t2 - t1).count();
    std::cout << "Sequential time: " << timeSeq << std::endl;

    t1 = std::chrono::high_resolution_clock::now();
    double res_std = stdMonteCarlo(Integral_1, { a }, { b }, N);
    t2 = std::chrono::high_resolution_clock::now();
    auto timePar = std::chrono::duration_cast<std::chrono::milliseconds>
        (t2 - t1).count();
    std::cout << "std time: " << timePar << std::endl;

    ASSERT_NEAR(res_seq, res_std, 0.5);
}

TEST(Monte_carlo_integral_test, test_result_of_integral_1) {
    std::vector<double> a = { 0.0, 2.5 };
    std::vector<double> b = { 1.534, 3.12 };
    int num_th = 4;
    const int N = 2500000;

    auto t1 = std::chrono::high_resolution_clock::now();
    double res_seq = seqMonteCarlo(Integral_2, a, b, N * num_th);
    auto t2 = std::chrono::high_resolution_clock::now();
    auto timeSeq = std::chrono::duration_cast<std::chrono::milliseconds>
        (t2 - t1).count();
    std::cout << "Sequential time: " << timeSeq << std::endl;

    t1 = std::chrono::high_resolution_clock::now();
    double res_std = stdMonteCarlo(Integral_2, a, b, N);
    t2 = std::chrono::high_resolution_clock::now();
    auto timePar = std::chrono::duration_cast<std::chrono::milliseconds>
        (t2 - t1).count();
    std::cout << "std time: " << timePar << std::endl;

    ASSERT_NEAR(res_seq, res_std, 0.5);
}

TEST(Monte_carlo_integral_test, test_result_of_integral_2) {
    std::vector<double> a = { 0.0, 2.5, 1.234 };
    std::vector<double> b = { 1.534, 3.12, 1.555 };
    int num_th = 4;
    const int N = 2500000;

    auto t1 = std::chrono::high_resolution_clock::now();
    double res_seq = seqMonteCarlo(Integral_3, a, b, N * num_th);
    auto t2 = std::chrono::high_resolution_clock::now();
    auto timeSeq = std::chrono::duration_cast<std::chrono::milliseconds>
        (t2 - t1).count();
    std::cout << "Sequential time: " << timeSeq << std::endl;

    t1 = std::chrono::high_resolution_clock::now();
    double res_std = stdMonteCarlo(Integral_3, a, b, N);
    t2 = std::chrono::high_resolution_clock::now();
    auto timePar = std::chrono::duration_cast<std::chrono::milliseconds>
        (t2 - t1).count();
    std::cout << "std time:" << timePar << std::endl;

    ASSERT_NEAR(res_seq, res_std, 0.005);
}

TEST(Monte_carlo_integral_test, test_result_of_integral_3) {
    std::vector<double>a = { 0.3, 1.32, 1.234 };
    std::vector<double>b = { 3.534, 3.5, 1.435 };
    int num_th = 4;
    const int N = 250000;

    auto t1 = std::chrono::high_resolution_clock::now();
    double res_seq = seqMonteCarlo(Integral_3, a, b, N * num_th);
    auto t2 = std::chrono::high_resolution_clock::now();
    auto timeSeq = std::chrono::duration_cast<std::chrono::milliseconds>
        (t2 - t1).count();
    std::cout << "Sequential time: " << timeSeq << std::endl;

    t1 = std::chrono::high_resolution_clock::now();
    double res_std = stdMonteCarlo(Integral_3, a, b, N);
    t2 = std::chrono::high_resolution_clock::now();
    auto timePar = std::chrono::duration_cast<std::chrono::milliseconds>
        (t2 - t1).count();
    std::cout << "std time: " << timePar << std::endl;

    ASSERT_NEAR(res_seq, res_std, 0.5);
}

TEST(Monte_carlo_integral_test, test_n_is_negative) {
    double a = 0.0, b = 3.0;
    ASSERT_ANY_THROW(stdMonteCarlo(Integral_1, { a }, { b }, -1000));
}

int main(int argc, char** argv) {
    ::testing::InitGoogleTest(&argc, argv);
    return RUN_ALL_TESTS();
}
\end{lstlisting}
\end{document}